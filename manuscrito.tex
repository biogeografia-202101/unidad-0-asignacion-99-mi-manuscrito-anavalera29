\documentclass[11pt,]{article}
\usepackage[left=1in,top=1in,right=1in,bottom=1in]{geometry}
\newcommand*{\authorfont}{\fontfamily{phv}\selectfont}
\usepackage[]{mathpazo}


  \usepackage[T1]{fontenc}
  \usepackage[utf8]{inputenc}



\usepackage{abstract}
\renewcommand{\abstractname}{}    % clear the title
\renewcommand{\absnamepos}{empty} % originally center

\renewenvironment{abstract}
 {{%
    \setlength{\leftmargin}{0mm}
    \setlength{\rightmargin}{\leftmargin}%
  }%
  \relax}
 {\endlist}

\makeatletter
\def\@maketitle{%
  \newpage
%  \null
%  \vskip 2em%
%  \begin{center}%
  \let \footnote \thanks
    {\fontsize{18}{20}\selectfont\raggedright  \setlength{\parindent}{0pt} \@title \par}%
}
%\fi
\makeatother




\setcounter{secnumdepth}{3}


\usepackage{graphicx,grffile}
\makeatletter
\def\maxwidth{\ifdim\Gin@nat@width>\linewidth\linewidth\else\Gin@nat@width\fi}
\def\maxheight{\ifdim\Gin@nat@height>\textheight\textheight\else\Gin@nat@height\fi}
\makeatother
% Scale images if necessary, so that they will not overflow the page
% margins by default, and it is still possible to overwrite the defaults
% using explicit options in \includegraphics[width, height, ...]{}
\setkeys{Gin}{width=\maxwidth,height=\maxheight,keepaspectratio}

\title{Título\\
Subtítulo\\
Subtítulo  }



\author{\Large Ana Hilda Valera Arias\vspace{0.05in} \newline\normalsize\emph{Estudiante, Universidad Autónoma de Santo Domingo (UASD)}  }


\date{}

\usepackage{titlesec}

\titleformat*{\section}{\normalsize\bfseries}
\titleformat*{\subsection}{\normalsize\itshape}
\titleformat*{\subsubsection}{\normalsize\itshape}
\titleformat*{\paragraph}{\normalsize\itshape}
\titleformat*{\subparagraph}{\normalsize\itshape}

\titlespacing{\section}
{0pt}{36pt}{0pt}
\titlespacing{\subsection}
{0pt}{36pt}{0pt}
\titlespacing{\subsubsection}
{0pt}{36pt}{0pt}





\newtheorem{hypothesis}{Hypothesis}
\usepackage{setspace}

\makeatletter
\@ifpackageloaded{hyperref}{}{%
\ifxetex
  \PassOptionsToPackage{hyphens}{url}\usepackage[setpagesize=false, % page size defined by xetex
              unicode=false, % unicode breaks when used with xetex
              xetex]{hyperref}
\else
  \PassOptionsToPackage{hyphens}{url}\usepackage[unicode=true]{hyperref}
\fi
}

\@ifpackageloaded{color}{
    \PassOptionsToPackage{usenames,dvipsnames}{color}
}{%
    \usepackage[usenames,dvipsnames]{color}
}
\makeatother
\hypersetup{breaklinks=true,
            bookmarks=true,
            pdfauthor={Ana Hilda Valera Arias (Estudiante, Universidad Autónoma de Santo Domingo (UASD))},
             pdfkeywords = {Género, Planta},  
            pdftitle={Título\\
Subtítulo\\
Subtítulo},
            colorlinks=true,
            citecolor=blue,
            urlcolor=blue,
            linkcolor=magenta,
            pdfborder={0 0 0}}
\urlstyle{same}  % don't use monospace font for urls

% set default figure placement to htbp
\makeatletter
\def\fps@figure{htbp}
\makeatother

\usepackage{pdflscape} \newcommand{\blandscape}{\begin{landscape}}
\newcommand{\elandscape}{\end{landscape}}


% add tightlist ----------
\providecommand{\tightlist}{%
\setlength{\itemsep}{0pt}\setlength{\parskip}{0pt}}

\begin{document}
	
% \pagenumbering{arabic}% resets `page` counter to 1 
%
% \maketitle

{% \usefont{T1}{pnc}{m}{n}
\setlength{\parindent}{0pt}
\thispagestyle{plain}
{\fontsize{18}{20}\selectfont\raggedright 
\maketitle  % title \par  

}

{
   \vskip 13.5pt\relax \normalsize\fontsize{11}{12} 
\textbf{\authorfont Ana Hilda Valera Arias} \hskip 15pt \emph{\small Estudiante, Universidad Autónoma de Santo Domingo (UASD)}   

}

}








\begin{abstract}

    \hbox{\vrule height .2pt width 39.14pc}

    \vskip 8.5pt % \small 

\noindent Resumen del manuscrito


\vskip 8.5pt \noindent \emph{Keywords}: Género, Planta \par

    \hbox{\vrule height .2pt width 39.14pc}



\end{abstract}


\vskip 6.5pt


\noindent  \section{Introducción}\label{introducciuxf3n}

La vegetación terrestre está constituida por un conjunto de plantas
pertenecientes a una familia en específico y esta a su vez se subdividen
en géneros y especies para identificarse dentro de su clase. Por
consiguiente no sería la excepción de la malvaceae, poseen 243 genéros y
más de 4,300 especies, sus flores son hermafroditas, pocas veces
unisexuales, solitarias o fasciculadas en las axilas de las hojas o
agrupadas en inflorescencia tal como la describen los siguientes autores
(Marín, Hilario, \& Andino, n.d.) y (Bayer, 2003).

Dentro de los géneros a encontrar en la familia malvaceae están el
abutilon constituido por arbustos, subarbustos y hierbas bienales con
pelo estrellados y tallos velloso, son carente del epicáliz conjunto de
apéndices que por lo regular tienen otros grupos de dicha familia, así
como de tener alrededor de 150 especie nativa en los trópicos y
subtrópicos de América, África, Asia y Australia, (Lorenzo-Cáceres,
2007). También, está el Hibiscus donde los segmentos del epicáliz estan
libres o unidos en la base, con estigmas alargados, semillas reniformes
y numerosas, (ORTIZ, 2010). Del mismo modo, se encuentra la althaea,
lavatera y la malva cada una contienen sus respectivas especies las
cuales pueden encontrarse en mayor o menor proporción en un espacio
determinado la cual dependerá de factores abioticos incidentes entre
ellos, lo que implicaría la necesidad de utilizar tecnicas y análisis
numerológicos para conocer su asocianción y distribción.

La implementación de análisis numéricos en las investigaciones
ecológicas permiten dar a conocer en terminos cuantificables la forma en
que se encuentran asociadas y el tipo de patrón que presenta algunas
especies, es por ello la importancia de la estratificación y
zonificación del objeto de estudio en cuestión. De acuerdo con
(González, 2006) esto permite conjugar en un mismo grupo información de
aquellos organismos que pueden ser cuantificable junto con otros que son
reproductivos y de manera general con toda la vegetación.

En tal sentido, este estudio por medio de la ecología numérica busca
conocer cómo estan asociados los diferentes géneros y especies de la
familia malvaceae y si las variables ambientales existente en la zona
influyen en dicha asociación. También, analizar como estan organizados
los grupos y qué patrón presentan en caso de que existiese y asimismo
establecer los indicadores ambientales que interfieren. De igual manera,
examinar en qué volumen se encuentran representadas cada una y distiguir
las especies alpha y beta, además, de construir modelos de distribución
espacial para idenficarla. Por consiguiente, esta investigación
constribuirá al conocimiento de la dinámica ecológica espacial que
envuelven las plantas pertenecientes a la familia malvaceae en la isla
de Barro Colorado que en lo adelante será llamado BCI y con la misma
gestionar estrategias para el cuidado y conservación de ésta.

\ldots

\section{Metodología}\label{metodologuxeda}

\subsection{Área de estudio}\label{uxe1rea-de-estudio}

La BCI se encuentra ubicada en el canal de Pánama en las proximidades
del lago Gatún, de acuerdo con (Pérez et al., 2005) esta se formó cuando
se construyó dicho canal embalsando las aguas del río Chagres, se
localiza entre las coordenadas geográficas (latitud 9\(^\circ\)~9'N,
longitud 79\(^\circ\)~ 50') y cubre una extensión de tierra de 1,500
hectáreas (ver figura \ref{mapa}). El clima es de bosque tropical, la
temperatura promedio es de 27 grados centígrados, con temporadas
lluviosas durante los meses mayo-diciembre y secas desde mediados de
diciembre hasta abril, las tormentas convectivas son prevenidas por los
vientos alisios dictando así las estaciones del año, (Sugasti, Eng, \&
Pinzón, 2018). Esta isla por sus caracteristicas fisicas sirve de
hábitat para muchos animales e insectos y por consiguiente para una
variedad de especie vegetal, convirtiendola en un espacio de
investigación de mucha importancia.

Es por ello, la escogencia como lugar de estudio la parcela permanente
de 50 hectárea de BCI donde se identificó como estan asociadas y
distribuidas la familia malvaceae a través del censo realizado por
(Hubbell, Condit, \& Foster, 2021) durante varios años (1981-1983 y
2010-2015, entre otros) donde se identificaron, marcaron y
cartografiaron todos los tallos leñosos independientes de al menos 10 mm
de diámetro a la altura.

\subsubsection{Materiales y Técnicas de
investigación}\label{materiales-y-tuxe9cnicas-de-investigaciuxf3n}

Para la realización de este estudio se utilizó el software de (R Core
Team, 2019) donde se cargaron varios paquetes de ecología numérica
suministrado por (Batlle, 2020) como el \emph{tidyverse} (dplyr) que
ayudó a formar matriz de comunidad que permitió identificar las
diferentes especies que abundan, en qué cantidad y orden de acuerdo a su
pH. También, el \emph{Simple Features} (sf) para crear área de hábitat
por cuadros y obtener la densidad de cada especie por metro cuadrado y
así conocer la abundancia y riqueza global. De igual manera, el
\emph{vegan} para caracterizar y analizar el orden y disimilaridades
entre cada especie y \emph{ez} para examinar las unidades o variables
repetitivas. Asimismo, el \emph{graphics}, \emph{psych} y \emph{mapvie}
para la representación grafica de cada datos.

\begin{figure}
\centering
\includegraphics{mapa_barro_colorado.jpeg}
\caption{Ubicación de la isla Barro Colorado\label{mapa}}
\end{figure}

\ldots

\section{Resultados}\label{resultados}

\ldots

\section{Discusión}\label{discusiuxf3n}

\section{Agradecimientos}\label{agradecimientos}

\section{Información de soporte}\label{informaciuxf3n-de-soporte}

\ldots

\section{\texorpdfstring{\emph{Script}
reproducible}{Script reproducible}}\label{script-reproducible}

\ldots

\section*{Referencias}\label{referencias}
\addcontentsline{toc}{section}{Referencias}

\hypertarget{refs}{}
\hypertarget{ref-jose_ramon_martinez_batlle_2020_4402362}{}
Batlle, J. R. M. (2020).
Biogeografia-master/scripts-de-analisis-BCI;coding sessions (Version
v0.0.9000). \url{https://doi.org/10.5281/zenodo.4402362}

\hypertarget{ref-bayer2003malvaceae}{}
Bayer, K., Clemens y Kubitzki. (2003). Malvaceae. In Springer (Ed.),
\emph{Plantas con flores ~textperiodcentered dicotiledóneas}.

\hypertarget{ref-gonzalez2006ecologia}{}
González, A. R. (2006). \emph{Ecología: Métodos de muestreo y análisis
de poblaciones y comunidades}. Pontificia Universidad Javeriana.

\hypertarget{ref-webcenso}{}
Hubbell, S., Condit, R., \& Foster, R. (2021). \emph{Parcela del censo
forestal en la isla de barro colorado}.

\hypertarget{ref-de2007especies}{}
Lorenzo-Cáceres, J. M. S. de. (2007). Las especies del género abutilon
mill.(Malvaceae) cultivadas en españa. \emph{PARJAP: Boletín de La
Asociación Española de Parques Y Jardines}, (45), 45--49.

\hypertarget{ref-marinanalisis}{}
Marín, J. Z., Hilario, R. F., \& Andino, O. O. (n.d.). \emph{Análisis
filogenético de la familia malvaceae}.

\hypertarget{ref-ortizclaves}{}
ORTIZ, D. G. (2010). \emph{Claves para los taxones y cultones del género
hibiscus l.(Malvaceae) cultivados y comercializados en la comunidad
valenciana (e españa)}.

\hypertarget{ref-perez2005metodologia}{}
Pérez, R., Aguilar, S., Condit, R., Foster, R., Hubbell, S., \& Lao, S.
(2005). Metodologia empleada en los censos de la parcela de 50 hectareas
de la isla de barro colorado, panamá. \emph{Centro de Ciencias
Forestales Del Tropico (CTFS) Y Instituto Smithsonian de Investigaciones
Tropicales (STRI)}, 1--24.

\hypertarget{ref-R}{}
R Core Team. (2019). \emph{R: A language and environment for statistical
computing}. Retrieved from \url{https://www.R-project.org/}

\hypertarget{ref-sugastimedicion}{}
Sugasti, L., Eng, B., \& Pinzón, R. (2018). \emph{Medición continúa de
flujo de co2 ensuelo en una parcela de bosque tropical en isla barro
colorado, canal de panamá.}




\newpage
\singlespacing 
\end{document}
