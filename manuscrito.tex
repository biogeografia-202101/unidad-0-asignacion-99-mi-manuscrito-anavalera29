\documentclass[11pt,]{article}
\usepackage[left=1in,top=1in,right=1in,bottom=1in]{geometry}
\newcommand*{\authorfont}{\fontfamily{phv}\selectfont}
\usepackage[]{mathpazo}


  \usepackage[T1]{fontenc}
  \usepackage[utf8]{inputenc}



\usepackage{abstract}
\renewcommand{\abstractname}{}    % clear the title
\renewcommand{\absnamepos}{empty} % originally center

\renewenvironment{abstract}
 {{%
    \setlength{\leftmargin}{0mm}
    \setlength{\rightmargin}{\leftmargin}%
  }%
  \relax}
 {\endlist}

\makeatletter
\def\@maketitle{%
  \newpage
%  \null
%  \vskip 2em%
%  \begin{center}%
  \let \footnote \thanks
    {\fontsize{18}{20}\selectfont\raggedright  \setlength{\parindent}{0pt} \@title \par}%
}
%\fi
\makeatother




\setcounter{secnumdepth}{3}


\usepackage{graphicx,grffile}
\makeatletter
\def\maxwidth{\ifdim\Gin@nat@width>\linewidth\linewidth\else\Gin@nat@width\fi}
\def\maxheight{\ifdim\Gin@nat@height>\textheight\textheight\else\Gin@nat@height\fi}
\makeatother
% Scale images if necessary, so that they will not overflow the page
% margins by default, and it is still possible to overwrite the defaults
% using explicit options in \includegraphics[width, height, ...]{}
\setkeys{Gin}{width=\maxwidth,height=\maxheight,keepaspectratio}

\title{Distribución espacial de los géneros y especies de la familia
\emph{Malvaceae} en una parcela de 50 ,ha. Caso: Isla Barro Colorado,
Panamá.  }



\author{\Large Ana Hilda Valera Arias\vspace{0.05in} \newline\normalsize\emph{Estudiante, Universidad Autónoma de Santo Domingo (UASD)}  }


\date{}

\usepackage{titlesec}

\titleformat*{\section}{\normalsize\bfseries}
\titleformat*{\subsection}{\normalsize\itshape}
\titleformat*{\subsubsection}{\normalsize\itshape}
\titleformat*{\paragraph}{\normalsize\itshape}
\titleformat*{\subparagraph}{\normalsize\itshape}

\titlespacing{\section}
{0pt}{36pt}{0pt}
\titlespacing{\subsection}
{0pt}{36pt}{0pt}
\titlespacing{\subsubsection}
{0pt}{36pt}{0pt}





\newtheorem{hypothesis}{Hypothesis}
\usepackage{setspace}

\makeatletter
\@ifpackageloaded{hyperref}{}{%
\ifxetex
  \PassOptionsToPackage{hyphens}{url}\usepackage[setpagesize=false, % page size defined by xetex
              unicode=false, % unicode breaks when used with xetex
              xetex]{hyperref}
\else
  \PassOptionsToPackage{hyphens}{url}\usepackage[unicode=true]{hyperref}
\fi
}

\@ifpackageloaded{color}{
    \PassOptionsToPackage{usenames,dvipsnames}{color}
}{%
    \usepackage[usenames,dvipsnames]{color}
}
\makeatother
\hypersetup{breaklinks=true,
            bookmarks=true,
            pdfauthor={Ana Hilda Valera Arias (Estudiante, Universidad Autónoma de Santo Domingo (UASD))},
             pdfkeywords = {Género, Planta},  
            pdftitle={Distribución espacial de los géneros y especies de la familia
\emph{Malvaceae} en una parcela de 50 ,ha. Caso: Isla Barro Colorado,
Panamá.},
            colorlinks=true,
            citecolor=blue,
            urlcolor=blue,
            linkcolor=magenta,
            pdfborder={0 0 0}}
\urlstyle{same}  % don't use monospace font for urls

% set default figure placement to htbp
\makeatletter
\def\fps@figure{htbp}
\makeatother

\usepackage{pdflscape} \newcommand{\blandscape}{\begin{landscape}}
\newcommand{\elandscape}{\end{landscape}}


% add tightlist ----------
\providecommand{\tightlist}{%
\setlength{\itemsep}{0pt}\setlength{\parskip}{0pt}}

\begin{document}
	
% \pagenumbering{arabic}% resets `page` counter to 1 
%
% \maketitle

{% \usefont{T1}{pnc}{m}{n}
\setlength{\parindent}{0pt}
\thispagestyle{plain}
{\fontsize{18}{20}\selectfont\raggedright 
\maketitle  % title \par  

}

{
   \vskip 13.5pt\relax \normalsize\fontsize{11}{12} 
\textbf{\authorfont Ana Hilda Valera Arias} \hskip 15pt \emph{\small Estudiante, Universidad Autónoma de Santo Domingo (UASD)}   

}

}








\begin{abstract}

    \hbox{\vrule height .2pt width 39.14pc}

    \vskip 8.5pt % \small 

\noindent Este estudio fue realizado en una parcela de 50 hectárea dentro de la
isla de Barro Colorado perteneciente al país de Pánama en Centroamérica
y tenía como finalidad conocer como estan asociadas, agrupadas y
distribuidas las especies de la familia \emph{Malvaceae}, asimismo, la
identificación de patrones, especies alpha y beta, además, de las
riquezas y abundancias de cada una de ellas. Los datos fueron obtenidos
a través de censos realizados en varios años y sometidos a diversos
software de información para el análisis, por medio de los cuáles se
determinó que en dicha familia existen especies que estan asociadas
dentro de ellas y en el espacio por variables ambientales y elementos
químicos. Siendo las zonas inclinadas, con acceso rápido al agua y
disponibilidad elevada de materia orgánica las más favorecidas. Formando
un patrón discontinuo en la agrupación de sus grupos, teniendo gran
concentración en sitios húmedos con presencia del \emph{zinc} (zn), el
\emph{Boro} (B) y pH, existiendo especies indicadoras como
\emph{Quararibea asterolepis} con índice de 0.978 \emph{Sterculia
apetala} con 0.914. Por otro lado, se estima una representación
significativa por cada área establecida de acuerdo al coeficiente de
correlación de Pearson con un índice de 0.6807189, en la cual por medio
del idicador de \emph{Hill} se encontró una distribución equitativa
entre las especies con un rango de riqueza de 5 a 13 y de abundancia de
31 a 127 por sitios, fungiendo la \emph{Quararibea asterolepis} como
principal contribuidora a la diversidad beta según el orden de
\emph{Hellinger} con 0.18005449.


\vskip 8.5pt \noindent \emph{Keywords}: Género, Planta \par

    \hbox{\vrule height .2pt width 39.14pc}



\end{abstract}


\vskip 6.5pt


\noindent  \section{Introducción}\label{introducciuxf3n}

La vegetación terrestre está constituida por un conjunto de plantas
pertenecientes a una familia en específico y esta a su vez se subdividen
en géneros y especies para identificarse dentro de su clase. Por
consiguiente no sería la excepción de la \emph{Malvaceae}, poseen 243
genéros y más de 4,300 especies, sus flores son hermafroditas, pocas
veces unisexuales, solitarias o fasciculadas en las axilas de las hojas
o agrupadas en inflorescencia tal como la describen los siguientes
autores (Marín, Hilario, \& Andino, n.d.) y (Bayer, 2003).

Dentro de los géneros a encontrar en la familia \emph{Malvaceae} están
el \emph{Abutilon} constituido por arbustos, subarbustos y hierbas
bienales con pelo estrellados y tallos velloso, son carente del epicáliz
conjunto de apéndices que por lo regular tienen otros grupos de dicha
familia, así como de tener alrededor de 150 especie nativa en los
trópicos y subtrópicos de América, África, Asia y Australia,
(Lorenzo-Cáceres, 2007). También, está el \emph{Hibiscus} donde los
segmentos del epicáliz estan libres o unidos en la base, con estigmas
alargados, semillas reniformes y numerosas, (ORTIZ, 2010). Del mismo
modo, se encuentra la \emph{Althaea}, \emph{Lavatera} y la \emph{Malva}
cada una contienen sus respectivas especies las cuales pueden
encontrarse en mayor o menor proporción en un espacio determinado la
cual dependerá de factores abioticos incidentes entre ellos, lo que
implicaría la necesidad de utilizar tecnicas y análisis numerológicos
para conocer su asocianción y distribción.

La implementación de análisis numéricos en las investigaciones
ecológicas permiten dar a conocer en terminos cuantificables la forma en
que se encuentran asociadas y el tipo de patrón que presenta algunas
especies, es por ello la importancia de la estratificación y
zonificación del objeto de estudio en cuestión. De acuerdo con
(González, 2006) esto permite conjugar en un mismo grupo información de
aquellos organismos que pueden ser cuantificable junto con otros que son
reproductivos y de manera general con toda la vegetación.

En tal sentido, este estudio por medio de la ecología numérica busca
conocer cómo estan asociados los diferentes géneros y especies de la
familia \emph{Malvaceae} y si las variables ambientales existente en la
zona influyen en dicha asociación. También, analizar como estan
organizados los grupos y qué patrón presenta en su distribución
espacial, asimismo, establecer los indicadores ambientales que
interfieren. De igual manera, examinar en qué volumen se encuentran
representadas cada una y distiguir los sitios con especies alpha y beta.
Por consiguiente, este estudio contribuirá al conocimiento de la
dinámica ecológica espacial que envuelven las plantas pertenecientes a
esta familia en la isla de Barro Colorado que en lo adelante será
llamado BCI y con la misma gestionar estrategias para el cuidado y
conservación de ésta.

\ldots

\section{Metodología}\label{metodologuxeda}

\subsection{Área de estudio}\label{uxe1rea-de-estudio}

La BCI se encuentra ubicada en el canal de Pánama en las proximidades
del lago Gatún, de acuerdo con (Pérez et al., 2005) esta se formó cuando
se construyó dicho canal embalsando las aguas del río Chagres, se
localiza entre las coordenadas geográficas (latitud 9\(^\circ\)~9'N,
longitud 79\(^\circ\)~ 50') y cubre una extensión de tierra de 1,500
hectáreas (ver figura \ref{mapa}). El clima se caracteriza por ser de
bosque tropical, la temperatura promedio es de 27 grados centígrados,
con temporadas lluviosas durante los meses mayo-diciembre y secas desde
mediados de diciembre hasta abril, las tormentas convectivas son
prevenidas por los vientos alisios dictando así las estaciones del año,
(Sugasti, Eng, \& Pinzón, 2018). Esta isla por sus caracteristicas
fisicas sirve de hábitat para muchos animales e insectos y por
consiguiente para una variedad de especie vegetal, convirtiendola en un
espacio de investigación de mucha importancia. Es por ello, la
escogencia como lugar de estudio la parcela permanente de 50 hectárea de
BCI, en la cual se identificó como estan asociadas y distribuidas la
familia \emph{Malvaceae} a través del censo realizado por (Hubbell,
Condit, \& Foster, 2021) durante varios años (1981-1983 y 2010-2015,
entre otros) donde se marcaron y cartografiaron todos los tallos leñosos
independientes de al menos 10 mm de diámetro de altura.

\begin{figure}
\centering
\includegraphics{mapa_barro_colorado.jpeg}
\caption{Ubicación de la isla Barro Colorado\label{mapa}}
\end{figure}

\subsubsection{Materiales y Técnicas de
investigación}\label{materiales-y-tuxe9cnicas-de-investigaciuxf3n}

Para la realización de este estudio se utilizó el software de (R Core
Team, 2019) donde se cargaron varios paquetes de ecología numérica como
el \emph{tidyverse} que ayudó a formar matriz de comunidad que permitió
identificar las diferentes especies que abundan, en qué cantidad y orden
de acuerdo a su pH. También, el \emph{Simple Features} (sf) para crear
área de hábitat por cuadros y obtener la densidad de cada especie por
metro cuadrado y así conocer la abundancia y riqueza global. De igual
manera, el \emph{Vegan} para caracterizar y analizar el orden y
disimilaridades entre cada especie y \emph{ez} para examinar las
unidades o variables repetitivas. Asimismo, el \emph{graphics},
\emph{psych} y \emph{mapvie} para la representación gráfica de cada
datos y (Kindt \& Coe, 2005) para señalar las especies alpha y beta,
cada \emph{script} utilizados fueron suministrado a partir del
repositorio de (Batlle, 2020) como fuente y el programa de información
geográfica Qgis (QGIS Development Team, 2009) para actualizar el mapa de
localización de la BCI.

\ldots

\section{Resultados}\label{resultados}

Por medio de los datos obtenidos a través del análisis de agrupamiento
al dividir el espacio en dos grupos uno con 42 y otro con 8 sitio, se
observó que existen 16 especies de la familia \emph{Malvaceae} con una
distancia muy corta dentro de la parcela de 50,ha. Siendo la
\emph{Quararibea asterolepis} la que más se asocia en el primero con
2,171 y la \emph{Sterculia apetala} en el segundo con 53 para un índice
de 0.978 y 0.914 respectivamente. En cuanto, al nivel de representación
y composición por cada 1,ha con la aplicación del método
\emph{Hellinger} los cuadros 33 hasta el 49 la dispariedad fue corta y
la similaridad numerosa, en tanto que, en el 8, 23, 30 y 35 fueron poco
similares y con intervalo largo. De igual manera, en la correlación de
diversidad, las riquezas y abundancia de especie fueron alto existiendo
una equidad en su distribución, como el del índice de \emph{Hill} aunque
las riquezas aumentan o disminuyen la ratio no son afectadas. También,
al aplicarse la prueba \emph{Moran's} se encontró que en la zona del
espolon,vaguada y vertiente la adecuación es de forma positiva,
asimismo, en los componentes químicos del \emph{calsio} (Ca),
\emph{cobre} (Cu), \emph{hierro} (Fe) y \emph{zinc} (Zn) y en el
piedemonte, el valle y la sima, al igual que en el \emph{manganeso} (Mn)
y \emph{aluminio} (Al) fue negativa. Por otro lado, el espacio 30 obtuvo
mayor riqueza en especie alpha con 13 y el más pobre el 45 con 5 con una
abundancia de 110 y 123 cosecutivamente, los que más varían el 13 y 46,
estando las especies \emph{Apeiba membranacea}, \emph{Apeiba tibourbou},
\emph{Hampea appendiculata}, \emph{Herrania purpurea} entre otras como
beta con un valor de 0.07936325, 0.06962371, 0.13708297, 0.10517566
sucesivamente (ver figura \ref{diversidad} \& \ref{ambiente}).

\begin{figure}
\centering
\includegraphics[width=1.00000\textwidth]{ecologia_espacial_files/figure-markdown_github/unnamed-chunk-10-1.png}
\caption{Correlación de las diferentes especies\label{diversidad}}
\end{figure}

\begin{figure}
\centering
\includegraphics[width=1.10000\textwidth]{matriz_correlacion_suelo_abun_riq_spearman.png}
\caption{Correlación de las diferentes especies\label{ambiente}}
\end{figure}

\begin{figure}
\centering
\includegraphics[width=1.10000\textwidth]{mapa_cuadros_abun_global.png}
\caption{Número de individuo por especie\label{riquezas}}
\end{figure}

\ldots

\section{Discusión}\label{discusiuxf3n}

La forma en que se encuentran asociados y distribuidos los géneros y
especies de la familia \emph{Malvaceae} en la parcela de 50,h de la BCI
se debe a factores abióticos como el tipo de formación de suelo,
determinado compuesto químico y por la presencia del pH los cuales
influyen de manera directa en su ordenación creando una dependencia
entre los grupos afectando su heterogeneidad. Aunque, el exceso de
ciertos elementos pueden afectar la distribución y crecimiento de
ciertas plantas (Clark, 2002), así, como el origen de formación del
suelo los cuales ofrecen ciertas propiedades que definirían el tipo de
colocación (Flores, Suvires, \& Dalmasso, 2015). En tal sentido, estos
factores ayudaron a formar patrones de agromeración en la zona media de
la isla, especialmente en la Vaguada, el espolón y la vertiente donde
los elementos como el \emph{zinc} (Zn), el \emph{boro} (B) y el
\emph{potasio} (K) fueron determinante, a diferencia de los espacios del
piedemonte, el valle, la sima y el conjunto de compuesto como el
\emph{manganeso} (Mn), \emph{cobre} (Cu) y el \emph{aluminio} (Al) donde
la correlación fue menor. De igual modo, se identificó las plantas
\emph{Quararibea asterolepis} y \emph{Sterculia apetala} como las que
más se asocian de manera combinada dentro de la familia y de manera
desigual en el espacio, esto posiblemente se deba a que fungen como
alimento o refugio para otros grupos (Alvarado-Hernández, 2011) las
cuales se encuentran dispersas por todo el terreno. Esta familia tiene
alto nivel de representación por cada 1,h, específicamente en los
bordes, entendiendo que sea el resultado por las condisiones climáticas
y a la disponibilidad del agua (Stevenson \& Rodríguez, 2008) existente
en esa área. Al mismo tiempo, de tener aportaciones de otras clases como
la \emph{Apeiba membranacea}, \emph{Hampea appendiculata} entre otras,
que contribuyen a la diversidad beta.

\section{Agradecimientos}\label{agradecimientos}

\section{Información de soporte}\label{informaciuxf3n-de-soporte}

\ldots

\section{\texorpdfstring{\emph{Script}
reproducible}{Script reproducible}}\label{script-reproducible}

\ldots

\section*{Referencias}\label{referencias}
\addcontentsline{toc}{section}{Referencias}

\hypertarget{refs}{}
\hypertarget{ref-alvarado2011caracterizacion}{}
Alvarado-Hernández, A. M. (2011). \emph{Caracterización floristica de
los hábitats utilizado por el tepezcuintle cuniculus paca,(LINNEO, 1766;
rodentia: Cuniculidae), en el piso basal del parque nacional carara,
costa rica}.

\hypertarget{ref-jose_ramon_martinez_batlle_2020_4402362}{}
Batlle, J. R. M. (2020).
Biogeografia-master/scripts-de-analisis-BCI;coding sessions (Version
v0.0.9000). \url{https://doi.org/10.5281/zenodo.4402362}

\hypertarget{ref-bayer2003malvaceae}{}
Bayer, K., Clemens y Kubitzki. (2003). Malvaceae. In Springer (Ed.),
\emph{Plantas con flores ~textperiodcentered dicotiledóneas}.

\hypertarget{ref-clark2002factores}{}
Clark, D. B. (2002). Los factores edáficos y la distribución de las
plantas. \emph{Ecología Y Conservación de Bosques Neotropicales. LUR,
Cartago, Costa Rica}, 193--221.

\hypertarget{ref-flores2015distribucion}{}
Flores, D. G., Suvires, G., \& Dalmasso, A. (2015). Distribución de la
vegetación nativa en ambientes geomorfológicos cuaternarios del monte
Árido central de argentina. \emph{Revista Mexicana de Biodiversidad},
\emph{86}(1), 72--79.

\hypertarget{ref-gonzalez2006ecologia}{}
González, A. R. (2006). \emph{Ecología: Métodos de muestreo y análisis
de poblaciones y comunidades}. Pontificia Universidad Javeriana.

\hypertarget{ref-webcenso}{}
Hubbell, S., Condit, R., \& Foster, R. (2021). \emph{Parcela del censo
forestal en la isla de barro colorado}.

\hypertarget{ref-biodiversidad}{}
Kindt, R., \& Coe, R. (2005). \emph{Tree diversity analysis. a manual
and software for common statistical methods for ecological and
biodiversity studies}. Retrieved from
\url{http://www.worldagroforestry.org/output/tree-diversity-analysis}

\hypertarget{ref-de2007especies}{}
Lorenzo-Cáceres, J. M. S. de. (2007). Las especies del género abutilon
mill.(Malvaceae) cultivadas en españa. \emph{PARJAP: Boletín de La
Asociación Española de Parques Y Jardines}, (45), 45--49.

\hypertarget{ref-marinanalisis}{}
Marín, J. Z., Hilario, R. F., \& Andino, O. O. (n.d.). \emph{Análisis
filogenético de la familia malvaceae}.

\hypertarget{ref-ortizclaves}{}
ORTIZ, D. G. (2010). \emph{Claves para los taxones y cultones del género
hibiscus l.(Malvaceae) cultivados y comercializados en la comunidad
valenciana (e españa)}.

\hypertarget{ref-perez2005metodologia}{}
Pérez, R., Aguilar, S., Condit, R., Foster, R., Hubbell, S., \& Lao, S.
(2005). Metodologia empleada en los censos de la parcela de 50 hectareas
de la isla de barro colorado, panamá. \emph{Centro de Ciencias
Forestales Del Tropico (CTFS) Y Instituto Smithsonian de Investigaciones
Tropicales (STRI)}, 1--24.

\hypertarget{ref-QGIS_software}{}
QGIS Development Team. (2009). \emph{QGIS geographic information
system}. Retrieved from \url{http://qgis.osgeo.org}

\hypertarget{ref-R}{}
R Core Team. (2019). \emph{R: A language and environment for statistical
computing}. Retrieved from \url{https://www.R-project.org/}

\hypertarget{ref-stevenson2008determinantes}{}
Stevenson, P. R., \& Rodríguez, M. E. (2008). Determinantes de la
composición florística y efecto de borde en un fragmento de bosque en el
guaviare, amazonía colombiana. \emph{Colombia Forestal}, \emph{11},
5--17.

\hypertarget{ref-sugastimedicion}{}
Sugasti, L., Eng, B., \& Pinzón, R. (2018). \emph{Medición continúa de
flujo de co2 ensuelo en una parcela de bosque tropical en isla barro
colorado, canal de panamá.}




\newpage
\singlespacing 
\end{document}
